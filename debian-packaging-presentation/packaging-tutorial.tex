\documentclass[red,10pt,a4paper]{beamer}
\usetheme{debian}
%\usepackage{beamerthemesplit}
%\usepackage{beamerthemeshadow}
%\setbeamercolor{background canvas}{bg=black}

%\setbeamercolor{normal text}{fg=white}

% Set slide numbers in footer
%\setbeamertemplate{footline}[frame number]

% Remove navigation symbols
%\setbeamertemplate{navigation symbols}{}

\usepackage{debiantutorial}



\usepackage[utf8]{inputenc}
\title{Debian Packaging Tutorial}
\subtitle{Magic that makes "\texttt{apt-get install}" work}
\author[Muneeb]{Muneeb Shaikh \\ \texttt{iammuneeb@gmail.com}}
\date{\today}

\begin{document}


% -- Title --
\begin{frame}
    \titlepage
\end{frame}




\begin{frame}{Outline}
  \tableofcontents[hideallsubsections]
\end{frame}

\section{General Installation Procedure}
\subsection{From Source}

\begin{frame}{From Source}
	\begin{itemize}
		\item Download the source from upstream
			\br
		\item Read the installation instructions
			\br
		\item Hunt for the pre-requisites of installing (Download Dependencies)
			\br
		\item Finally install with these commands
			\begin{enumerate}
				\item \texttt{./configure}
					\hbr
				\item \texttt{make}
					\hbr
				\item \texttt{make install}
			\end{enumerate}
	\end{itemize}
\end{frame}

\subsection{From Repository}

\begin{frame}{From Repository}
\begin{center}
\large{\texttt{sudo apt-get install package\_name}}
\end{center}
\end{frame}

\section{Packaging}

\subsection{Tools of Trade}
\begin{frame}{Tools of Trade}
  \begin{itemize}
  \item A Debian (or Ubuntu) system (with root access)
    \br
  \item Some packages:
    \begin{itemize}
    \item \textbf{build-essential}: contains basic building tools such as 
    \textbf{gcc, g++, make} and mainly \textbf{dpkg-dev}, which contains basic
        Debian-specific tools to create packages
     
      \hbr
    \item \textbf{devscripts}: contains many useful scripts for Debian maintainers
    \hbr
    \item \textbf{dh-make}: tool to Debianize the upstream source easily
    \hbr
    \item \textbf{lintian}: Debian package checker
    \end{itemize}
  \end{itemize}
\end{frame}

\subsection{General packaging workflow}
\begin{frame}{General packaging workflow}
  \begin{center}
    \begin{tikzpicture}[
      node1/.style={shape=rectangle,draw=rouge,fill=debianbackgroundblue,thick},
      arr/.style={very thick}, command/.style={text=rouge,font=\ttfamily}, ]
      
      \node[node1] (www) at (0, 0) {Web};
      \node[node1] (us) at (2.5, 0) {upstream source};
      \node[node1] (da) at (-2.5, 0) {Debian mirror};
      \node[node1] (sp) at (0, -2) {source package};
      \draw[arr,<-,dashed,thick] (sp) -- (2.5,-2) node[right=0cm,text width=2.98cm,text centered,font=\small\sl] {where most of the manual work is done};
      \node[node1] (bin) at (0, -4) {one or several binary packages};
      \draw[arr,<-,dashed,thick] (bin) -- (3.5,-4) node[right,text centered,font=\small\ttfamily\sl] {.deb\normalfont};
      \draw[arr,->] (us) -- (sp) node[pos=0.5,right,command] {dh\_make};
      \draw[arr,->] (da) -- (sp) node[pos=0.5,left,command] {apt-get source};
      \draw[arr,->] (www) -- (sp) node[pos=0.5,left,command] {dget};
      \draw[arr,->] (sp) -- (bin) node[pos=0.5,right,text width=6cm] {\textttc{debuild} (build and test with \textttc{lintian}) or \textttc{dpkg-buildpackage}};
      \draw[arr,->] (bin) -- (1,-6) node[pos=0.5,right] {install (\textttc{debi})};
      %	\draw[arr,->] (bin) -- (-1,-6) node[pos=0.5,left] {upload (\textttc{dput})};
      \draw[transparent] (bin) -- (-1,-6) node[pos=0.5,left,opaque] {upload (\textttc{dput})};
      \draw[arr,->,rounded corners] (bin) -- (-1,-6) -- (-4.5,-6) -- (-4.5,0) -- (da);
      \useasboundingbox (-4,-6) rectangle (6,0); % hack hack hack
    \end{tikzpicture}
  \end{center}
\end{frame}


\section{Creating Debian Package Steps}
\subsection{Steps}

\begin{frame}{Creating Debian Package Steps}
\begin{enumerate}
	\item Download the upstream tarball
	\item Rename the upstream tarball
	\item Unpack the upstream tarball
	\item Add the Debian packaging files
	\item Build the package
	\item Check for errors
	\item Install the package
	\item If everything is working as expected upload it to \structure{\url{mentors.debian.net}}
\end{enumerate}

\end{frame}

\subsection{Step 1: Download upstream tarball}

\begin{frame}[fragile]{Step 1: Download upstream tarball}
\lstset{breaklines=true}
\lstset{makemacrouse=true}
\begin{lstlisting}

$ mkdir ~/packaging
$ cd ~/packaging
$ wget http://download.savannah.gnu.org/releases/smc/hyphenation/patterns/hyphen-as-0.7.0.tar.bz2
    \end{lstlisting} 
\end{frame}


\begin{frame}[fragile]{The Deb package format}
  \begin{itemize}
  \item \texttt{.deb} file: an \texttt{ar} archive
    \begin{lstlisting}[basicstyle=\ttfamily\footnotesize]
$ ar tv wget_1.12-2.1_i386.deb
rw-r--r-- 0/0      4 Sep  5 15:43 2010 debian-binary
rw-r--r-- 0/0   2403 Sep  5 15:43 2010 control.tar.gz
rw-r--r-- 0/0 751613 Sep  5 15:43 2010 data.tar.gz
    \end{lstlisting} % $
    \begin{itemize}
    \item \texttt{debian-binary}: version of the deb file format, \texttt{"2.0\textbackslash{}n"}
    \item \texttt{control.tar.gz}: metadata about the package\\
      {\small \texttt{\textbf{control}, md5sums, (pre|post)(rm|inst), triggers, shlibs}, \ldots}
    \item \texttt{data.tar.gz}: data files of the package
    \end{itemize}
  \end{itemize}

\end{frame}
























\end{document}